\section{Strongly Typed Actors}

Strongly typed actors provide a convenient way of defining type-safe messaging interfaces.
When calling \lstinline^become^ in a strongly typed actor, \emph{all} message handlers from the typed interface must be set.

Typed actors use handles of type \lstinline^typed_actor<...>^ rather than \lstinline^actor^, whereas the template parameters hold the messaging interface.
For example, an actor responding to two integers with a dobule would use the type \lstinline^typed_actor<replies_to<int, int>::with<double>>^.
All functions for message passing, linking and monitoring are overloaded to accept both types of actors.

\subsection{Spawning Typed Actors}
\label{sec:strong:spawn}

Typed actors are spawned using the function \lstinline^spawn_typed^.
The most convenient way to is to use a typedef for a particular kind of actor.

\begin{lstlisting}
// assign to identical type
using my_actor = typed_actor_ptr<
                   replies_to<int, int>::with<double>,
                   replies_to<double, double>::with<double, double>
                 >;
my_actor::behavior_type my_actor_impl() {
  [](int a, int b) {
   return static_cast<double>(a) * b;
  },
  [](double a, double b) {
    return std::make_tuple(a * b, a / b);
  }
);
auto p0 = spawn_typed(my_actor_impl);
my_actor p1 = p0;
// assign to subtype
using subtype1 = typed_actor_ptr<
                   replies_to<int, int>::with<double>
                 >;
subtype1 p2 = p0;
// assign to another subtype
using subtype2 = typed_actor_ptr<
                   replies_to<double, double>::with<double, double>
                 >;
subtype2 p3 = p0;
\end{lstlisting}

\clearpage
\subsection{Class-based Typed Actors}

Typed actors are spawned using the function \lstinline^spawn_typed^ and define their message passing interface as list of \lstinline^replies_to<...>::with<...>^ statements.
This interface is used in (1) \lstinline^typed_event_based_actor<...>^, which is the base class for typed actors, (2) the handle type \lstinline^typed_actor<...>^, and (3) \lstinline^typed_behavior<...>^, i.e., the behavior definition for typed actors.
Since this is rather redundant, the actor handle provides definitions for the behavior as well as the base class, as shown in the example below.
It is worth mentioning that all typed actors always use the event-based implementation, i.e., there is no typed actor implementation providing a blocking API.

\begin{lstlisting}
struct shutdown_request { };
struct plus_request { int a; int b; };
struct minus_request { int a; int b; };

typedef typed_actor<replies_to<plus_request>::with<int>,
                    replies_to<minus_request>::with<int>,
                    replies_to<shutdown_request>::with<void>>
        calculator_type;

calculator_type::behavior_type
typed_calculator(calculator_type::pointer self) {
    return {
        [](const plus_request& pr) {
            return pr.a + pr.b;
        },
        [](const minus_request& pr) {
            return pr.a - pr.b;
        },
        [=](const shutdown_request&) {
            self->quit();
        }
    };
}

class typed_calculator_class : public calculator_type::base {
 protected: behavior_type make_behavior() override {
        return {
            [](const plus_request& pr) {
                return pr.a + pr.b;
            },
            [](const minus_request& pr) {
                return pr.a - pr.b;
            },
            [=](const shutdown_request&) {
                quit();
            }
        };
    }
};
\end{lstlisting}

\clearpage
\begin{lstlisting}
void tester(event_based_actor* self, const calculator_type& testee) {
    self->link_to(testee);
    // will be invoked if we receive an unexpected response message
    self->on_sync_failure([=] {
        aout(self) << "AUT (actor under test) failed" << endl;
        self->quit(exit_reason::user_shutdown);
    });
    // first test: 2 + 1 = 3
    self->sync_send(testee, plus_request{2, 1}).then(
        [=](int r1) {
            assert(r1 == 3);
            // second test: 2 - 1 = 1
            self->sync_send(testee, minus_request{2, 1}).then(
                [=](int r2) {
                    assert(r2 == 1);
                    // both tests succeeded
                    aout(self) << "AUT (actor under test) "
                               << "seems to be ok"
                               << endl;
                    self->send(testee, shutdown_request{});
                }
            );
        }
    );
}

int main() {
    // announce custom message types
    announce<shutdown_request>();
    announce<plus_request>(&plus_request::a, &plus_request::b);
    announce<minus_request>(&minus_request::a, &minus_request::b);
    // test function-based impl
    spawn(tester, spawn_typed(typed_calculator));
    await_all_actors_done();
    // test class-based impl
    spawn(tester, spawn_typed<typed_calculator_class>());
    await_all_actors_done();
    // done
    shutdown();
    return 0;
}
\end{lstlisting}
